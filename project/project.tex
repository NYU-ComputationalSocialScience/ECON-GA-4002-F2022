\documentclass[11pt]{article}

\oddsidemargin=0.25truein \evensidemargin=0.25truein
\topmargin=-0.5truein \textwidth=6.0truein \textheight=8.75truein

\usepackage{hyperref}
\urlstyle{rm}   % change fonts for url's (from Chad Jones)
\hypersetup{
    colorlinks=true,        % kills boxes
    allcolors=blue,
    pdfsubject={ECON-GA-4002 Project},
    pdfauthor={CCSLTS},
    pdfstartview={FitH},
    pdfpagemode={UseNone},
}

%\renewcommand{\thefootnote}{\fnsymbol{footnote}}

% table layout
\usepackage{booktabs}

% section spacing and fonts
\usepackage[small,compact]{titlesec}

% list spacing
\usepackage{enumitem}
\setitemize{leftmargin=*, topsep=0pt}
\setenumerate{leftmargin=*, topsep=0pt, partopsep=0pt}

% document starts here
\begin{document}
\parskip=\bigskipamount
\parindent=0.0in
\thispagestyle{empty}

\bigskip\bigskip
\centerline{\Large \bf ECON-GA-4002:  Project Guide}
\centerline{Revised: \today}

\section*{Overview}

One key outcome of this course is for you to produce a piece of work you can use
to demonstrate your computational skills. That work will come in the form of an
interactive notebook. These notebooks allow you to combine code, text, and
graphics in one user-friendly document. We will add them to our GitHub repository
so that you can use a link to show others what you've done. You will also be able
to see what your classmates have done.

The relatively loose structure in self-directed projects like this makes them
more challenging than most things you do in school. We think that also makes
them more interesting. They give you a chance to indulge your curiosity and show
off your creativity.

We have divided the project into components to keep you on track. The intent is
to make the project easier by breaking it down into a number of small,
manageable sub-projects. The early steps are graded only on whether you do them:
you get full points if you do them, none if you don't. We start with individual
work.  Partway along, we encourage you to form groups of up to two, but if
you'd prefer to do this on your own, that's ok, too.


\section*{Project outline}

The components of the project are
%
\begin{center}
\begin{tabular}{lllll}
\toprule
Assignment                  & Format  & Individual/Group &  Points & Due Date\\
\midrule
Three Project Ideas         & Paper (hand-written) & Individual & 5 & 15 November \\
Project Proposal            & Paper (professional) & Group & 10 & 5 December \\
Final Project               & Pluto/Jupyter notebook & Group & 85 & 19 December \\
\bottomrule
\end{tabular}
\end{center}

The components consist of:
\begin{itemize}

\item {\bf Three project ideas.}

Write down three project ideas in class. One or two sentences each is enough.
Use your imagination.  Be creative.  Speak to others. Write down things that
interest you. This will not be graded, but it will give you something to work
with later on.

Over the coming weeks, we recommend you keep your ears open for possible
{\bf project ideas}

\item{\bf Project proposal.}

If you so desire, form a group of up to two people --- no more --- and choose
a single project from those you submitted --- or perhaps some other idea if you
get a sudden flash of inspiration.

Flesh out the project in more detail, and two figures you plan to produce with
it. Total length should be no more than two pages.

\item {\bf Final project.}

You should submit your {\bf notebook\/} to \href{mailto:sgl290@nyu.edu}{sgl290@nyu.edu} and
\href{mailto:cgc332@nyu.edu}{cgc332@nyu.edu} by the due date listed on the course website.

The subject line should be:  ``ECON-GA-4002 Project''. The
file name should be your last names separated by dashes and a short title ---
something like {\tt Jones-Smith-Zhang-India.ipynb}.

Your project should include:

\begin{itemize}
  \item Description cell.  Put a Markdown cell at the top of your notebook that
  includes the title of your project, a list of authors, and a short summary of
  what you do. Think of the last one as an advertisement to potential readers.
  \item Graphics. A series of figures that tell us something interesting. Three
  or four would be enough, but do what you think works best for your project.
  Think about the narrative:  What story do you want to tell?
\end{itemize}

\end{itemize}


\section*{Free advice}

Some things to keep in mind:
%
\begin{itemize}
\item {\bf Keep it simple.}

Most project ideas turn out to be too big.  You're generally well-advised to
carve out a manageable subset of what you think you can do. There's no reason to
worry about this at the idea generation stage, but as you develop your project
you may find that you need to focus more narrowly on a part of it.

\item {\bf Ask for help.}
We have years of experience with this kind of thing. If you're stuck, let us
know and we'll try to help. You can also post questions on the class discussion
board.

\end{itemize}


\section*{Grading}

Projects will be graded on their overall quality.  This includes, but is not
restricted to, these categories:
%
\begin{itemize}
\item Quality of the idea.  Is the question clearly articulated?  Is it
interesting? Does it have general appeal?
\item Degree of difficulty. Some ideas are harder than others to implement. As
in Olympic diving, you get credit for taking on a challenge.
\item Professional look.  Does your project look professional?  Are the graphs
easy to understand?  Are they clearly labeled?
\item Demonstration of skills. Does your project showcase what you've learned in
this class? Do you use some of the theory or algorithms we've worked on? How
does your application of these tools impact your final writeup?
\end{itemize}



%\end{document}
\pagebreak
\thispagestyle{empty}

\bigskip\bigskip
\centerline{\Large \bf ECON-GA-4002:  Project Grade Sheet}
\centerline{Revised: \today}


\section*{Overall}

\begin{itemize} %\itemsep=0.75\bigskipamount
\item Clear idea and message?
\item Documentation of sources?
\item Effective graphics?
\item Professional appearance?
\end{itemize}
\bigskip

\section*{Idea}

\vspace*{0.75in}

%\begin{itemize} \itemsep=2\bigskipamount
%\item Clarity of the message
%\item ..
%\end{itemize}

\section*{Graphics}

\vspace*{0.75in}

%\begin{itemize} \itemsep=2\bigskipamount
%\item Clarity of the message
%\item ..
%\end{itemize}

\section*{Difficulty}

\vspace*{0.75in}

%\begin{itemize} \itemsep=2\bigskipamount
%\item Clarity of the message
%\item ..
%\end{itemize}

\section*{Professional look}

\vspace*{0.75in}

%\begin{itemize} \itemsep=2\bigskipamount
%\item Clarity of the message
%\item ..
%\end{itemize}


\end{document}
