%\documentclass[11pt]{article}

%%%%%%%%%%%%%%%%%%%%%% Stock Preamble %%%%%%%%%%%%%%%%%%%%%%%%%%%%%%%%%%
\documentclass[12pt,pdftex,twoside,letterpaper]{exam}

\usepackage{amsmath,amssymb, amsthm}
\usepackage{graphicx}
\usepackage{color}
\usepackage{comment}
\usepackage{layout}
\usepackage{booktabs}
\usepackage[flushleft]{threeparttable}
\usepackage{caption}
\usepackage{setspace}
\usepackage{float}
\usepackage{needspace}
\usepackage[colorlinks=true,linkcolor=blue,citecolor=black,urlcolor=blue,bookmarks=false,pdfstartview={FitV}]{hyperref}

%%%%%%%%%%%%%%%%%%%%%%Margins%%%%%%%%%%%%%%%%%%%%%%%%%%%%%%%%%%%%%%%%%%%%%%%
\usepackage[margin=1.0in]{geometry}
\setlength{\parindent}{0in}
\setlength{\parskip}{.09in}
\raggedbottom

%%%%%%%%%%%%%%%%%%%%Tighten up the lists%%%%%%%%%%%%%%%%%%%%%%%%%%%%%%%%%%
\let\OLDdescription\description
\renewcommand\description{\OLDdescription\setlength{\itemsep}{-2mm}}

%%%%%%%%%%%%%%%%%%%%%%%%%Exam class formating%%%%%%%%%%%%%%%%%%%%%%%%%%%%%%%%%%%%%%%%
\renewcommand{\partlabel}{\thepartno.}
\renewcommand{\questionshook}{\setlength{\itemsep}{0.2in}}
\renewcommand{\partshook}{\setlength{\leftmargin}{0.2in}}
\renewcommand\familydefault{\sfdefault}

%%%%%%%%%%%%%%%%%%%%%%%%%%%%%%%%%Let Backus fight the good fight, I'm going with LN%%%%%%%%%%%%%%%%%%%
\renewcommand{\log}{\ln}

%%%%%%%%%%%%%%%%%%%%%%Headers and Footers%%%%%%%%%%%%%%%%%%%%%%%%%%%%%%%%%%
\pagestyle{headandfoot}

\runningheadrule
\firstpageheadrule
\runningheader{Math}{}{Syllabus: ECON-GA 4002}
\runningfooter{}{}{}

\begin{document}

  \centerline{\Large\bf Syllabus: Mathematical Foundations for CSS}
  \vspace{3mm}
  \centerline{\large\bf ECON-GA 4002 $|$ Fall 2022}
  \vspace{3mm}
  \centerline{\bf Revised: \today}

  \bigskip

  This course focuses on the mathematical foundations necessary to understand and master the tools
  that define modern ``computational social science''. This course will be different than many of
  the math classes that you have taken previously in the sense that we will use computers and the
  Python programming language to develop an understanding of the tools and to demonstrate how we
  can apply the mathematical tools we cover. This does not mean that we will not discuss the
  underlying theory, but it does mean that we will try to approach topics from a more practical
  standpoint.

  We will use Python, a popular high-level computer language, that is being used widely across many
  fields. ``High-level'' means it's less painful than most (the hard work is done by the language),
  but it's a serious language with extensive capabilities.

  This course will lay the foundations for additional topics that will be covered in subsequent
  semesters and is an important foundation for machine learning and data science.

  \subsubsection*{About the instructors}

    This course is co-taught by three instructors: Chase Coleman (cgc332@nyu.edu), Spencer Lyon
    (sgl290@nyu.edu), and Thomas Sargent (ts43@nyu.edu).

    Office hours: By appointment

  \subsubsection*{Where and When}

    Meeting times: Tuesday 6:00 pm - 8:50 pm (Eastern Standard Time)

    Meeting place: gather.town -- See brightspace for the link to join class

  \section*{Requirements}

  There are minimal prerequisites to our course. You will benefit from having taken a certain
  amount of mathematics but the only programming that we will assume is what was covered in the
  online summer pre-course. We welcome students who have no prior programming experience (beyond
  what was covered in the pre-course) and have tried to design the course with them in mind. That
  said, the coures will require the {\bf courage} to take on a challenge and the {\bf patience} to
  fix computer programs that don't work.

  Our one requirement, is that students should expect to be on their laptop or computer during
  class so that they can follow along with the code that we're discussing and write small blocks
  of code on their own to further understanding.

  \section*{Getting help}

    This course is meant to be collaborative and to have a strong support system to help you when
    you run into problems. Please use the discussion board or other forms of communication to
    communicate with your classmates and ask questions/help each other.

    The bottom line:  {\bf If you're stuck, ask for help\/}.

    Really.  Don't be a hero, ask for help.

  \section*{Course materials and assignments}

    All course materials will be posted on \href{Github}{https://github.com/NYU-ComputationalSocialScience/ECON-GA-4002}.

    We will keep a running log of what we do in class using the REAME file that displays on the
    repository's main page.

    Assignments, including your exam, will be assigned through Github Classroom.

  \section*{Deliverables and grades}

    Graded work includes:

    \begin{itemize}
      \item {\bf Course participation.} We expect students to participate in class. If you have
          questions, don't be shy about asking them. The discussion forums are a great place to
          participate in class as well.
      \item {\bf Code Practice.} There will be several homework assignments throughout the
          semester. We encourage you do to all of them (they're good practice) but your lowest
          grade will dropped. We find that people who finish these assignments tend to keep up
          with the material better and these are easy points to get in terms of grades.
      \item {\bf Quizzes.} There will be a quiz most days that we have class. These quizzes should
          take between 5 and 10 minutes and will test general Python knowledge and review
          material from the previous class.
      \item {\bf Exams.} There will be two exams. These exams will be take home and you may use
          your notes and the internet. The only exception is that we'd ask that you don't ask
          the exam questions in online forums and that you don't speak with your classmates about
          the exam.
      \item {\bf Project.} The main deliverable for the semester will be a project. We will discuss
          what this project entails in a separate document.
    \end{itemize}


    {\bf Due dates} will be posted on the Brightspace.

    {\bf Dates are not negotiable. Anything handed in late will get a grade of zero.}

    {\bf All your work should be clean and professional.} We expect your math to be written in
    LaTeX for this course and expect your code to conform to good code habits.

    {\bf Final grades\/} will be computed from

    \begin{center}
      \begin{tabular}{ll}
        Participation / Professionalism & 5\% \\
        Code practice & 15\% \\
        Quizzes & 15\% \\
        Exams & 25\% \\
        Project & 40\%
      \end{tabular}
    \end{center}

    The weighting reflects our opinion that the most important skills to be acquired in this class
    are communicated by one's ability to successfully apply the tools that you learn to an
    interesting question in the social sciences.

    Final grades are not subject to any fixed distribution or curve. The number of A grades, for
    example, will depend only on your performance in the course.

\section*{Recommended work habits}

  Python is not something you can learn from reading a book and attending lectures. You need to
  {\bf write programs}... The more the better. Think about how you'd learn to play basketball
  or soccer; reading and listening to lectures aren't enough, you need to do it. We'll do a lot of
  programming in class, but it's {essential} that you follow up outside of class.

\section*{Pacing}

  The course is designed to be cover material at whatever pace the class is capable of. The topics
  should take roughly one week each, but we can scale that up or down as needed. If you're an expert,
  don't worry, we'll cover a lot of material either way.

\section*{Other questions}

  We encourage students who have questions to typically post their questions on the Brightspace
  site so that answers can be referenced by other students. If no answer is provided in a reasonable
  amount of time (i.e. wait at least 24 hours), you may email us to remind us of the question. If you
  have a question about a matter that should be kept private, please don't hesitate to reach out
  directly by email.

  \newpage
\section*{Schedule and weekly learning goals}

  The schedule is tentative and subject to change. Several of the modules below will occupy more than one week.  The learning goals target
  key concepts to be mastered after each module. Successive modules build on early modules.

\begin{center}
\begin{tabular}{lrl}
Date & Class & Topic\\
\hline
2022-09-06 Tue & 1 & Linear Algebra\\
2022-09-13 Tue & 2 & Probability Theory\\
2022-09-20 Tue & 3 & Probability and Statistics \\
2022-09-27 Tue & 4 & Probability Distributions \\
2022-10-04 Tue & 5 & GEometric Series, Sameulson Model\\
2022-10-11 Tue & - & No Class (Monday Schedule)\\
2022-10-18 Tue & 6 & Convex Optimization 1 \\
2022-10-25 Tue & 7 & Convex Optimization 2\\
2022-11-01 Tue & 8 & Convex Optimization 3\\
2022-11-08 Tue & 9 & Convex Optimization 4\\
2022-11-15 Tue & 10 & Statistics 1\\
2022-11-22 Tue & 11 & Statistics 2\\
2022-11-23 -- 2022-11-27  & -- & Thanksgiving Holiday\\
2022-11-28 Tue & 12 & Bayesian Statistics 1\\
2022-12-05 Tue & 13 & Bayesian Statistics 2\\
2022-12-12 Tue & 14 & Project Presentations\\
\end{tabular}
\end{center}


\newpage
\section*{Policies}

\begin{itemize}
  \item \textbf{General Behavior.} The School expects that students will conduct themselves with
        respect and professionalism toward faculty, students, and others present in class and will
        follow the rules laid down by the instructor for classroom behavior.  Students who fail to
        do so may be asked to leave the classroom.

  \item \textbf{Collaboration on Graded Assignments.} You may discuss assignments with anyone
        (in fact, we encourage it), but anything you submit, including your code, should be your
        own. Exams should be entirely your own work. Violation of this policy will result in a
        failing grade for the course.

  \item \textbf{Academic Integrity.}

  \begin{itemize}
    \item Integrity is critical to the learning process and to all that we do here at NYU. As
          members of our community, all students agree to abide by the NYU Student Code of Conduct,
          which includes a commitment to:
    \item Exercise integrity in all aspects of one's academic work including, but not limited to,
          the preparation and completion of exams, papers and all other course requirements by not
          engaging in any method or means that provides an unfair advantage.
    \item Clearly acknowledge the work and efforts of others when submitting written work as one's
          own. Ideas, data, direct quotations (which should be designated with quotation marks),
          paraphrasing, creative expression, or any other incorporation of the work of others
          should be fully referenced.
    \item Refrain from behaving in ways that knowingly support, assist, or in any way attempt to
          enable another person to engage in any violation of the Code of Conduct. Our support also
          includes reporting any observed violations of this Code of Conduct or other School and
          University policies that are deemed to adversely affect the NYU community.
  \end{itemize}

  The entire Student \href{https://www.nyu.edu/about/policies-guidelines-compliance/compliance/code-of-ethical-conduct.html}{Code of Conduct}
  applies to all students enrolled in NYU courses. \textbf{Any violation of the a policies
  pertaining to Academic Integrity will result in a failing grade for the course.}
\end{itemize}

\subsubsection*{Students with disabilities}

  If you have a qualified disability that requires academic accommodation during this course,
  please contact the Moses Center for Students with Disabilities (CSD, 212-998-4980) and ask them
  to send me a letter verifying your registration and outlining the accommodation they recommend.
  If you need to take an exam at the CSD, you must submit a completed Exam Accommodations Form to
  them at least one week prior to the scheduled exam time to be guaranteed accommodation.

\end{document}
